\chapter{Load-Store-Unit (LSU)}

The LSU of the core takes care of accessing the data memory. Load and stores on
words (32 bit), half words (16 bit) and bytes (8 bit) are supported.

Table~\ref{tab:lsu_signals} describes the signals that are used by the LSU.


\begin{table}[H]
 \caption{LSU Signals}
 \label{tab:lsu_signals}
  \begin{tabularx}{\textwidth}{@{}llX@{}} \toprule
    \textbf{Signal}               & \textbf{Direction} & \textbf{Description} \\ \toprule
    \signal{data\_req\_o}         & \textbf{output}    & Request ready, must stay high until \signal{data\_gnt\_i} is high for one cycle \\ \hline
    \signal{data\_addr\_o[31:0]}  & \textbf{output}    & Address \\ \hline
    \signal{data\_we\_o}          & \textbf{output}    & Write Enable, high if we want to write, low if we want to read \\ \hline
    \signal{data\_be\_o[3:0]}     & \textbf{output}    & Byte Enable, is set for the bytes to write/read \\ \hline
    \signal{data\_wdata\_o[31:0]} & \textbf{output}    & Data to be written to memory \\ \hline
    \signal{data\_rdata\_i[31:0]} & \textbf{input}     & Data read from memory \\ \hline
    \signal{data\_rvalid\_i}      & \textbf{input}     & \signal{data\_rdata\_i} is valid. This signal must always be identical to \signal{data\_gnt\_i} delayed by one cycle. \\ \hline
    \signal{data\_gnt\_i}         & \textbf{input}     & The memory accepted the request and will answer in the next cycle with valid rdata \\ \bottomrule
  \end{tabularx}
\end{table}

\section{Misaligned Accesses}
The LSU is able to perform misaligned accesses, meaning accesses that are not
aligned on natural word boundaries. However it needs to perform two separate
word-aligned accesses internally.
This means that at least two cycles are needed for misaligned loads and stores.


\section{Post-Increment Load and Stores}

Post-incrementing load and store instructions perform a load/store operation
from/to the data memory while at the same time increasing the base address by
the specified offset.
Post-incrementing load and stores reduce the number of instructions necessary to
execute when running in a loop, i.e. the address increment can be embedded in
the post-increment instructions.


\section{Protocol}
\label{sec:lsu_protocol}

The protocol that is used by the LSU to communicate with a memory works as
follows:
The LSU provides a valid address in \signal{data\_addr\_o} and sets
\signal{data\_req\_o} high. The memory then answers with a \signal{data\_gnt\_i}
set high as soon as it is ready to serve the request. This may happen in the
same cycle as the request was sent or any number of cycles later. After a grant
was received, the address may be changed by the LSU without impact. Also the
\signal{data\_wdata\_o}, \signal{data\_we\_o} and \signal{data\_be\_o} signals
may be changed as it is assumed that the memory has already processed that
information. In the case of a read, the memory answers with a
\signal{data\_rvalid\_i} set high when \signal{data\_rdata\_i} is valid. This
may happen one cycle after the grant was received, but may take any number of
cycles after the grant was received.
Starting from the cycle when \signal{data\_rvalid\_i} was asserted, another
request may be sent.

Figure~\ref{fig:lsu_trans_basic}, Figure~\ref{fig:lsu_trans_b2b} and
Figure~\ref{fig:lsu_trans_slow} show timing diagrams of the protocol.

\begin{figure}[H]
  \centering
  \begin{tikztimingtable}
    [timing/d/background/.style={fill=white},
     timing/lslope=0.1,
     xscale=3]

    clk              & 13{C} \\
    data\_addr\_o    & UUU 4D{Address} UU   UU UU \\
    data\_wdata\_o   & UUU 4D{WData} UU     UU UU \\
    data\_req\_o     & LLL HH HH  LL        LL LL \\
    data\_gnt\_i     & LLL LL HH  LL        LL LL \\
    data\_rvalid\_i  & LLL LL LL  HH        LL LL \\
    data\_rdata\_i   & UUU UU UU  2D{RData} UU UU \\
    data\_we\_o      & UUU 4D{WE} UU        UU UU \\
    data\_be\_o      & UUU 4D{BE} UU        UU UU \\
  \end{tikztimingtable}
  \caption{Basic Memory Transaction}
  \label{fig:lsu_trans_basic}
\end{figure}


\begin{figure}[H]
  \centering
  \begin{tikztimingtable}
    [timing/d/background/.style={fill=white},
     timing/lslope=0.1,
     xscale=3]

    clk              & 13{C} \\
    data\_addr\_o    & U 2D{Addr1}  2D{Addr2}  UU UU UU UU \\
    data\_wdata\_o   & U 2D{WData1} 2D{Wdata2} UU UU UU UU \\
    data\_req\_o     & L HH         HH         LL LL LL LL \\
    data\_gnt\_i     & L HH         HH         LL LL LL LL \\
    data\_rvalid\_i  & L LL         HH         HH LL LL LL \\
    data\_rdata\_i   & U UU         2D{RData1} 2D{RData2} UU UU UU\\
    data\_we\_o      & U 2D{WE1}    2D{WE2}    UU UU UU UU \\
    data\_be\_o      & U 2D{BE1}    2D{BE2}    UU UU UU UU \\
  \end{tikztimingtable}
  \caption{Back to Back Memory Transaction}
  \label{fig:lsu_trans_b2b}
\end{figure}

\begin{figure}[H]
  \centering
  \begin{tikztimingtable}
    [timing/d/background/.style={fill=white},
     timing/lslope=0.1,
     xscale=3]

    clk              & 13{C} \\
    data\_addr\_o    & U 6D{Address} UU   UU        UU \\
    data\_wdata\_o   & U 6D{WData} UU     UU        UU \\
    data\_req\_o     & L HH HH HH  LL        LL        LL \\
    data\_gnt\_i     & LLL LL HH  LL        LL        LL \\
    data\_rvalid\_i  & LLL LL LL  LL        HH        UU \\
    data\_rdata\_i   & UUU UU UU  UU        2D{RData} UU \\
    data\_we\_o      & U 6D{WE}   UU        UU        UU \\
    data\_be\_o      & U 6D{BE}   UU        UU        UU \\
  \end{tikztimingtable}
  \caption{Slow Answer Memory Transaction}
  \label{fig:lsu_trans_slow}
\end{figure}
