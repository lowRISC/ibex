\chapter{Pipeline}
\label{chap:pipeline}

\rvcore has a fully independent pipeline, meaning that whenever possible data
will propagate through the pipeline and therefor does not suffer from any
unneeded stalls.

The pipeline design is easily extendable to incorporate out-of-order
completion. E.g. it would be possible to complete an instruction that only
needs the EX stage before the WB stage, that is currently blocked waiting for
an rvalid, is ready.
Currently this is not done in \rvcore, but might be added in the future.

Figure~\ref{fig:pipeline} shows the control signals relevant for the pipeline
operation. Running from right to left are the main control signals, the
\signal{ready} signals of each pipeline stage.
Each pipeline stage has two control inputs: an enable and a clear. The enable
activates the pipeline stage and the core moves forward by one instruction. The
clear removes the instruction from the pipeline stage as it is completed.
At every pipeline stage, when the \signal{ready} coming from the stage to the
right is high, but the valid signal of the stage is low, the stage is cleared.
If the valid signal is high, it is enabled.

Going from right to left every stage is independent, no stage depends on the
stage on its left. Going from left to right every stage depends on its right
neighbor and can only continue when it is ready.
This means that in addition that a single stage has to be ready, also the stage
on its right has to be ready to move on.

\begin{figure}[H]
  \centering
  \includegraphics[width=0.9\textwidth]{./figures/pipeline}
  \caption{\rvcore Pipeline.}
  \label{fig:pipeline}
\end{figure}



\begin{boxnote}
In contrast to \orion there is no global stall controller any more, instead
every stage manages its own dependencies. The main controller of \rvcore is now
only responsible for control flow operations, i.e. flushing the pipe, branches,
jumps and exceptions.
\end{boxnote}
